\chapter{Conclusion and Future Work}

The fast increasing global urbanization process in recent years has greatly changed the world in the recent 50 years. The human migration from the rural area to the urban area results in a series of problems such as environment pollutant, traffic jam and insufficient local resource supply. Meanwhile, with the development of science and technology, a variety of data source can be collected easier and cheaper, which also provide the opportunity for analysts to understand the phenomenon and solve the problems. Although a lot of automatic techniques have been proposed in recent years, most of them still have limitations such as the lack of interpretability, difficulty in handling unexpected cases. Due to the variety of urban applications, these techniques always generate uncontrollable results which is hard to be accepted by the non-expert users. Thus the human experience and knowledge are required to make judgement or provide guidance in the analyzing process. Visual analytics, integrating the intuitive graphics and auto techniques, bridge the gap between users and complex urban data by taking human into the analysis loop.  

This proposal consists of three works that cover the three domains of urban information: place, people and technology. We start with a brief summary of the existing visualization techniques used in urban data exploration. After that, we introduced three visual analytics system target at the human-scale urban form exploration, large scale movement visualization and urban computing technique interpretation. 

In Chapter 2, we introduce Streetvizor, a framework enabling urban analysts to explore the human scale urban from city-, region- and street-level. The proposed system takes the street view images as input and extracts human-scale urban form features by leveraging the advanced machine learning techniques. Then, a visual analytics system is built upon the urban form feature database. The proposed method is evaluated by the real world data from four cities: London, Singapore, Hong Kong, and New York City. 

In Chapter 3, we developed RAEB(route aware edge bundling), a novel edge bundling technique to visualize the overview of a large amount of OD trails in urban traffic data. Target at the limitation of traditional edge bundling techniques, we introduce the urban traffic network as a constraint to the trail bundles. A series of new parameters, together with adaptions of existing ones, are employed in the pipeline.

In Chapter 4, we propose MultiRNNExplorer, a visual analytics system which helps analysts to understand the RNNs in high-dimensional time series forecast. The visual analytics system aims to interpret RNNs from two aspects, the overall model mechanism and the feature importance. Several case studies on air pollutant forecast show the effectiveness of the proposed technique.

Even though several novel designs and comprehensive analytics systems have been discussed in this proposal, the current development of urban related visualization techniques still cannot meet the analysis requirement due to the increasing complexity and variety of the applications. Based on the current work, we list several promising future directions.

\textbf{High-level urban analysis based on street view images.} In Chapter 2, we discuss how to leverage the easily collected street view images and advanced machine learning techniques to help us build the comprehensive visual analytics system for human-scale urban form explorations. In the future, we will extend the current framework to the more complex tasks. For example, an exiting research work uses vehicle classification techniques to estimate the demographic makeup of the US; We can also extract the signboards and building style from the street view images to support the exploration of urban region  functionalities.  

\textbf{Improve the bundling and rendering efficiency for the large trajectory dataset.} In Chapter 3, the edge bundling and rendering process still require tens of seconds to finish the visualization tasks, which makes the interactive exploration impossible. Our proposed technique is built upon the kernel density estimation techniques which can be parallelized. In the future, we will use the GPU-acceleration to improve bundling efficiency. In addition, to improve the computing efficiency, another strategy is to reduce the data amount. Traditional sampling methods, such as uniform and stratified sampling, cannot keep visualization accuracy. We plan to develop visualization aware sampling techniques to support the fast OD trails bundling. 

\textbf{Support model ensemble exploration.} Due to the complexity and variety of urban information exploration, a single forecast technique is always not enough to produce a credible and accurate result. In the future, we will extend the technique interpretation from RNNs to technique ensembles. However, use the same criteria to interpret multiple techniques is difficult due to the different algorithm mechanisms. Thus, we plan to figure out some uniform properties of these techniques (such as the feature importance) to make the them comparable. 