\chapter{Conclusion and Future Work}

The fast increasing global urbanization process in recent years has greatly changed the world in the recent 50 years. The human migration from the rural area to the urban area results in a series of problems such as environment pollutant, traffic jam and insufficient local resource supply. Meanwhile, with the development of science and technology, a variety of data source can be collected easier and cheaper, which also provide the opportunity for analysts to understand the phenomenon and solve the problems. Although a lot of automatic techniques have been proposed in recent years, most of them still have limitations such as the lack of interpretability, difficulty in handling unexpected cases. Due to the variety of urban applications, these techniques always generate uncontrollable results which is hard to be accepted by the non-expert users. Thus the human experience and knowledge are required to make judgement or provide guidance in the analyzing process. Visual analytics, integrating the intuitive graphics and auto techniques, bridge the gap between users and complex urban data by taking human into the analysis loop.  

This proposal consists of three works that cover the three domains of urban information: place, people and technology. We start with a brief summary of the existing visualization techniques used in urban data exploration. After that, we introduced three visual analytics system target at the human-scale urban form exploration, large scale movement visualization and urban computing technique interpretation. 

In Chapter 2, we introduce Streetvizor, a framework enabling urban analysts to explore the human scale urban from city-, region- and street-level. The proposed system takes the street view images as input and extracts human-scale urban form features by leveraging the advanced machine learning techniques. Then, a visual analytics system is built upon the urban form feature database. The proposed method is evaluated by the real world data from four cities: London, Singapore, Hong Kong, and New York City. 

In Chapter 3, we developed RAEB(route aware edge bundling), a novel edge bundling technique to visualize the overview of a large amount of OD trails in urban traffic data. Target at the limitation of traditional edge bundling techniques, we introduce the urban traffic network as a constraint to the trail bundles. A series of new parameters, together with adaptions of existing ones, are employed in the pipeline.

