\chapter{Visual Exploration of Human-Scale Urban Forms Based on Street Views}\label{chap:intro}
\QM{Urban forms at human-scale, i.e., urban environments that individuals can sense (e.g., sight, smell, and touch) in their daily lives, can provide unprecedented insights on a variety of applications, such as urban planning and environment auditing.
The analysis of urban forms can help planners develop high-quality urban spaces through evidence-based design.
However, such analysis is complex because of the involvement of spatial, multi-scale (i.e., city, region, and street), and multivariate (e.g., greenery and sky ratios) natures of urban forms.
In addition, current methods either lack quantitative measurements or are limited to a small area. 
The primary contribution of this work is the design of StreetVizor, an interactive visual analytics system that helps planners leverage their domain knowledge in exploring human-scale urban forms based on street view images.
Our system presents two-stage visual exploration: 
1) an AOI Explorer for the visual comparison of spatial distributions and quantitative measurements in two areas-of-interest (AOIs) at city- and region-scales;
2) and a Street Explorer with a novel parallel coordinate plot for the exploration of the fine-grained details of the urban forms at the street-scale.
We integrate visualization techniques with machine learning models to facilitate the detection of street view patterns. 
We illustrate the applicability of our approach with case studies on the real-world datasets of four cities, i.e., Hong Kong, Singapore, Greater London and New York City.
Interviews with domain experts demonstrate the effectiveness of our system in facilitating various analytical tasks.}

\section{Introduction}
%%%%%%%%%%%%%%%%%%%%%%%%%%%%%%%%%%%%%%%%%%%%%%%%%%%%%%%%%%%%%%%%%%%%%%%%
Human-scale urban form describes fine-scale characteristics of urban environments that can be directly seen, touched, and experienced by a city's residents in their daily lives~\cite{long_2016_human-scale}.
It is typically measured in high-resolution by sight and hearing, i.e., from several to tens of meters.
Compared with a city's scale, which is usually measured in kilometers, this scale is human-oriented.
 % and allows people to interact with surrounding environments.
% For instance, views on two neighboring streets can be totally different, even though the streets are very close to each other.
As humans pay more attention to interactive surroundings~\cite{gehl_1971_life}, understanding human-scale urban forms is essential for urban planners in designing high-quality urban spaces. 
However, traditional urbanism theories, such as small-scale surveys and mapping, are hard to provide in-depth guidance for effective urban planning and design at this fine scale.

Given the advancement of various sensing technologies, e.g., cameras and GPS devices, we can now quantitatively measure human-scale urban forms by analyzing big urban data.
In particular, services, such as Google Street View (GSV)~\cite{anguelov2010google}, provide detailed panoramic views of urban space from different geographic positions
These panoramic views can be utilized to measure various features, including greenery coverage and sky visibility, of human-scale urban forms visible to human eyes.
Some pioneering studies have shown that neighborhood environment~\cite{rundle_2011_using}, street-level greenery~\cite{li_2015_accessing}, and even street safety~\cite{Naik_2014_streetscore} can be precisely assessed from these views.
 % through analyzing GSV images.

However, GSV image exploration mainly focuses on either a particular feature (e.g., greenery coverage~\cite{li_2015_accessing}) or a small area (e.g., neighborhood~\cite{rundle_2011_using} and street~\cite{Naik_2014_streetscore, li_2015_accessing}).
This deficiency limits its applicability in urban planning, where planners need to 
1) quantitatively measure multivariate features of urban forms, including not only greenery coverage, but also sky visibility, and vehicle density~\cite{long_2017_how}; 
2) systematically explore urban forms in areas-of-interest (AOIs) at multiple scales, i.e., from small (e.g., streets) to mid (e.g., districts) to large scales (e.g., cities)~\cite{liu_2015_understanding}.
In addition, direct means for the comparison of urban forms in two AOIs is desirable to allow planners to utilize information for the quick identification and improvement of factors that affect the quality of urban space.

A visual analytics tool is necessary to fulfill these requirements because it can integrate powerful computing capabilities to quantitatively measure multivariate features, with interactive visual interfaces to systematically explore and compare features in AOIs on demand~\cite{sun_2013_survey}.
Developing such a tool requires considerable effort because of the following reasons:
first, as cities comprise vast amounts of street views, an efficient feature extraction algorithm is required to automatically uncover human-scale urban forms.
Second, the development of a tool for the visual comparison of multivariate features in two AOIs requires an effective visual design that tackles the challenges of spatial, multivariate, and comparative data visualizations.

In this paper, we introduce StreetVizor, a visual analytics system for the exploration of human-scale urban forms based on GSV images.
We develop the system in an iterative design process: specific analysis requirements are described by a collaborating urban planner, and the designs are evaluated and refined against requirements.  
To present information in concisely, StreetVizor combines a set of well-established visualization techniques, including coordinated multiple views (CMVs) and scatterplot matrix, with a new design of parallel coordinates that integrate street layout information.
Our system utilizes advanced clustering models to enable the efficient exploration of street view patterns.
We apply StreetVizor in real-world datasets containing $\sim$1.7 million of GSV images of four cities: Singapore, Hong Kong, Greater London, and New York City, and demonstrate its effectiveness through interviews with domain experts. 

\vspace*{2mm}
The main contributions of this work include:

\begin{itemize}
	
\vspace*{-1.5mm}
\item
A fully automatic approach measuring human-scale urban forms by applying deep learning techniques on GSV images;
	
\vspace*{-1.5mm}
\item
A visual comparison framework for exploring human-scale urban forms on city-, region-, and street-scales;
	
\vspace*{-1.5mm}
\item
A novel visual design of parallel coordinates that integrate street layout information;
	
\vspace*{-1.5mm}
\item
Interesting insights revealed from case studies and expert interviews, such as the negative correlation between $greenery$ and $building$ features, and the differences in street views of two cities.
		
\end{itemize}

% \vspace*{-1mm}
\section{Related Work}
\label{sec:related_work}

This section discusses previous studies closely related to our work.

%===============================================
\subsection{Street View Analysis}
GSV system provides high quality and accurate panoramic images of hundreds of cities~\cite{anguelov2010google}.
In recent years, researchers studying human-scale urban forms have utilized GSV images as a new and convenient data source. 
For example, researches have shown that the analysis of GSV images can be used to audit neighborhood environments~\cite{rundle_2011_using}, quantify street greenery~\cite{li_2015_accessing}, and predict street safety~\cite{Naik_2014_streetscore}.
Nonetheless, the majority of these studies face scalability issues given their focus on either a particular feature~\cite{li_2015_accessing} or a small area~\cite{rundle_2011_using, Naik_2014_streetscore, li_2015_accessing}.
Thees issues can be addressed by incorporating deep learning techniques, which can be used to summarize city landscapes~\cite{doersch2015makes} and estimate the demographic makeup of a country~\cite{gebru2017using}.

In this work, we collect $\sim$1.7 million GSV images of four representative mega-cities, and apply a deep learning technique~\cite{Badrinarayanan_2015_segnet} to automatically extract desired urban forms from the collected images.
More importantly, we develop an effective visual analytics tool for urban planners to explore human-scale urban forms.

%===============================================
\subsection{Urban Data Visualization}
Vast amounts of urban data, including traffic~\cite{ferreira_visual_2013, wang_2013_visual}, social media~\cite{xu_2013_visual, chen_2015_interactive}, environment~\cite{ferreira_2011_birdvis}, and simulated urban spaces~\cite{vanegas_2009_visualization}, have been collected in an urban context.
Big urban data brings in unprecedented opportunities for evidence-based urban design, and visualization systems can assist domain experts in finding evidence from the data.
A systematic overview of visualization systems can be found in~\cite{zheng_2016_visual}.

Qu et al.~\cite{qu_2007_visual} presented a comprehensive visualization system for the analysis of a city's air pollution that affects the daily lives of residents.
Their system integrates parallel coordinates and scatterplots to show relationships between high-dimensional air pollutants.
In addition to air pollution, landmark visibility is related to the daily experience of a city's residents.
Ortner et al.~\cite{ortner_2016_vis-a-ware} visually compared the effects of candidate buildings on landmark visibility from various viewpoints.
In this system, users can select a series of ranking schemes, and candidate buildings are then automatically sorted.
Similar to our present work, Arietta et al.~\cite{arietta_2014_city} associated visual elements with city attributes, including violent crime rates and housing prices.
They developed various prototype visualizations, such as the visual boundary of urban neighborhoods.

% \vspace{2mm}
Although different data are explored, these visualizations similarly employ CMVs, because urban data typically exhibits both spatial information and multi-dimensional attributes.
Our system also adopts this empirical approach.
In addition, to address specific domain problems, we develop effective visualization techniques, including a novel parallel coordinates enhanced with street layouts.

%===============================================
\subsection{Multivariate Geographical Data Visualization}
Visualizing multivariate data is a hot topic in the visualization field.
Numerous conventional approaches to this topic have been developed, and can be classified into two groups:
1) employing visualization techniques, such as parallel coordinates plot (PCP), scatterplot matrix, and start coordinates;
and 2) projecting data points onto a two- or three-dimensional visual space that can be directly plotted on a screen, such as multidimensional scaling and principal component analysis.
All these approaches have pros and cons.
For example, although PCP presents all dimensional attributes without information loss, it can easily generate visual clutter with big data and pairwise correlations can only be shown on two nearby coordinates~\cite{heinrich_2012_state}.
Many improvements have been developed to address these issues.
These improvements include edge bundling to reduce visual clutter~\cite{zhou2008visual, holten_2010_evaluation}, and hierarchical data clustering and the navigation of resulting structures~\cite{fua1999hierarchical, zhao_2012_structure}.
% ., and sort dimensions in order based on their correlations~\cite{zhao_2012_structure, wu_2015_telcovis}.

When multivariate data is dependent upon locations, the analytical tasks become more complex because geographical information needs to be revealed.
Turkay et al. developed \textit{Attribute Signature}~\cite{turkay_2014_attribute}, which employs a geographical map and small multiples of multivariate attributes to show geographic variability in attribute statistics.
Goodwin et al.~\cite{goodwin_2016_visualizing} further explored multivariate geographical data across scales by adopting new designs to show correlation, scale, and geographical information.
The frameworks proposed by both studies can be generalized to explore multivariate geographical data.

In this work, human-scale urban forms to be explored are also multivariate geographical data: the features are in six dimensions and they are dependent on locations.
We leverage the advantages of scatterplot matrix and PCP for different analytical tasks.
Specifically, we employ scatterplot matrix for exploring features at city- and region scales given that it can effectively reveal correlations between all pair-wise features.
We also arrange the views in a way similar to \textit{Attribute Signature}~\cite{turkay_2014_attribute}, i.e., geographical information is presented on maps and multivariate attributes in small multiples. 
In addition, we develop a novel PCP enhanced with a themeriver plot, which fits better with the analytical task of showing feature variations along street layout at street-scale.

%===============================================
\subsection{Comparative Visualization}
Gleicher et al.~\cite{gleicher_2011_visual} classified techniques for visual comparison into three categories:
1) Juxtaposition, i.e., presenting objects next to each other. 
For example, NodeTrix~\cite{yang_2017_blockwise} arranges two human brain networks side-by-side. 
2) Superposition, i.e., presenting multiple objects on top of one another.
Typical examples are time-series line graphs that plot the changes in several variables over time in the same coordinate system.
3) Explicit encoding, i.e., presenting differences or correlations between objects visually.
For instance, the bivariate density map is employed in~\cite{zeng_2017_visualizing} to show the relationship between departure and arrival movements over space.
In practice, these techniques are combined to address complex analytical tasks.

Our work adopts juxtaposition that arranges maps of two AOIs/streets side-by-side (Fig.~\ref{fig:teaser}(b) \& (d)), and superposition to compare multivariate features of two AOIs/streets in the same coordinate system (Fig.~\ref{fig:teaser}(c) \& (e)).

\if 0
\subsection{Human-scale Urban Form}
Though human-scale urban form is defined recently~\cite{long_2016_human-scale}, discussions of the concept in urban planning and designing have a long history that can be traced to 1960s.
A series of pioneering studies~\cite{jacobs_1961_life, gehl_1971_life} claimed the positive effects of human-scale urban form in designing high-quality urban space.
Among various types of human-scale urban form, the visible one is particularly important as human beings tend to pay most attentions to surroundings that can be directly seen~\cite{gehl_1971_life}. 

Accompanying with the raising of big open data, e.g., GSV images, researches in studying visible human-scale urban form are focusing on quantitatively measuring the related features nowadays. 
For example, researches have shown that analyzing GSV images can audit neighborhood environments~\cite{rundle_2011_using}, quantify street greenery~\cite{li_2015_accessing} and predict street safety~\cite{Naik_2014_streetscore}.
Nonetheless, most of these researches got scalability issues as they focused on either a particular feature or a small area.
The issues can be addressed by incorporating deep learning techniques, by which recent studies have shown its applicability in summarizing landscape of a city~\cite{doersch2015makes} and even estimating demographic makeup of a country~\cite{gebru2017using}.

\vspace{2mm}
In this work, we first identify the important features of visible human-scale urban form, then collect millions of GSV images across four representative mega-cities, and apply a deep learning technique - SegNet~\cite{Badrinarayanan_2015_segnet} - to automatically label related features in the images.
More importantly, we develop an effective visual analytical tool for urban planners to explore the analysis on demand.

\fi

\if 0
\subsection{Street view images analysis}

Google street view(GSV) is a well know system that provides panoramic images in hundreds of cities of more than 20 countries to millions of googles users. During the past 15 year, GSV system can provide a good quality and accurate images service, and more and more researchers begin to focus on this data source\cite{anguelov2010google}. GSV is considered as a novel and convenient way to collect environmental information and has been utilized in a broad range of research domains including ecology, geography, archeology, urbanology and even some social science. 

In recent years, the researchers in the ecology field found that the GSV images was an alternative way to observe the natural environment. Olea et al. \cite{olea2013assessing} discussed that the GSV images could be a good data source to assess the habit of some cliff living animals and the using of both of these two methods could be highly useful as a coarse-scale assessment method over large geographic areas. Hardion et al.\cite{hardion2016species} proposed a  time and cost-effective solution using the geo-located street view images in the analysis of species distributions. In the discipline of urbanology and sociology, GSV has been used in the virtual audit of different physical environment\cite{clarke2010using, rundle2011using, kelly2013using, vanwolleghem2014assessing} in the city. These methods has the potential to significantly reduce the costs and time of collecting data and can be well adapted in the large geo-scale research. 

Even though the GSV images could be adapted in different research areas, in the most instance, the observation of environment is still conducted by human beings. Nowadays, fast developed of deep learning techniques has been widely used in image processing, many research has considered the automatic methods to further reduce the human cost. Doersch et al.\cite{doersch2015makes} analyzed millions of image from GSV to extract the visual features commonly occurred to summary the landscape of a city. Timnit et al.\cite{gebru2017using} also proposed a method use deep learning techniques and GSV images to estimate teh demographic makeup of United States.
\fi


\section{Background and Analytical Tasks}
In this section, we introduce our research background and summarize the desired analytical tasks.

\subsection{Background}
\label{sec:bg}

Although the concept of human-scale urban form has only been recently defined~\cite{long_2016_human-scale}, its discussions in the context of urban planning has a long history that can be traced back to the 1960s.
A series of pioneering studies~\cite{jacobs_1961_life, gehl_1971_life} claimed the positive effects of understanding human-scale urban forms in designing high-quality urban space.
Visible human-scale urban forms are particularly important as human beings tend to pay most attentions to surroundings that can be directly seen~\cite{gehl_1971_life}. 

Over the past 10 months, we closely worked with a senior researcher ($SR$) in the field of evidence-based urban design $-$ an emerging research topic in urban planning and design.
$SR$ pointed out that though urban planners have begun to realize the importance and usefulness of street views in analyzing visible urban forms (e.g.,~\cite{rundle_2011_using, li_2015_accessing, Naik_2014_streetscore}), systematic and efficient methods that can facilitate exploration remain lacking.
Hence, $SR$ proposed the development of an efficient visual analytics tool for exploring human-scale urban forms based on GSV images.

To better understand the problem domain, we conducted several rounds of structured interviews with $SR$.
The main analysis criteria are summarized below:

\vspace*{-2mm}
\begin{itemize}

\item
\textbf{Multivariate Features.}
As images contain rich information on the urban environment, the first step is to identify the urban forms for analysis.
Here, we identify five key features that can reflect the quality and livability of street spaces~\cite{jacobs_1961_life, gehl_1971_life}, i.e., $greenery$, $sky$, $building$, $road$, and $vehicle$ features.
$Greenery$ reflects the pleasing greenery view of a street; 
$sky$ and $building$ are correlated with the sense of street closure negatively and positively, respectively; 
and increments in $road$ and $vehicle$ ratios decrease the willingness of people to walk and street attractiveness.

\vspace*{-1mm}
\item
\textbf{Street View Crawling.}
To reveal the surrounding scenes of a street space, street views have to be crawled appropriately: successive images should reflect the continuous change in surrounding scenes.
Hence, the distance between two successive views should not exceed a limit that produces discontinuous scenes; meanwhile, it should not be too small, which will cause computing overload. 
After experimenting with several options, we find 50 meters is a suitable value for the distance between two successive views.

\vspace*{-1mm}
\item
\textbf{Street View Directions.}
Although GSV~\cite{gsv_api} provides 360-degree panorama imagery, only the front and back images in the directions of street headings at sampling locations are required. 
Side views are not utilized because of the following considerations: 
First, side views mainly present building facades and street sides and thus cannot correctly reflect other key features of street space, e.g., $road$.
Second, side views are partially contained by the front and back images at nearby sampling locations. 

\end{itemize}

\vspace*{-1mm}
To evaluate the effectiveness of our approach, we first experiment with a few representative cities.
$SR$ suggested Hong Kong, Singapore, Greater London, and New York City:
Hong Kong and Singapore are dense cities with high-rise buildings in Asia.
Greater London and New York City are well-planned cities in Europe and the US.

%===============================================
\subsection{Analytical Tasks}
\label{ssec:tasks}

After identifying the analysis criteria, $SR$ further raised a list of questions for our system to address, including:
\textit{How are the identified features distributed in an AOI?
What are the feature differences between two AOIs?
What are the exact views that people can see on a street?
Are there any representative views?}

Based on these questions, we compile a list of analytical tasks:

\vspace*{-2mm}
% \begin{enumerate}[label={T.\arabic*:}]
\begin{itemize}

\item \textbf{\QM{T1. Efficient Multi-scale Exploration:}}
Human-scale urban forms are associated with street views at different locations that can be organized on city-, region- and street-scales.
Planners first need an intuitive overview of the identified feature distributions within a city or a region (T.1.1).
Next, planners need to explore the details of the urban forms, such as the exact street views, at street level (T.1.2).
Effective interactions are required to assist users in navigating across different scales.

\vspace*{-1.5mm}
\item \textbf{T2. Quantitative Measurements:}
% Traditionally, planners usually evaluate a street space qualitatively based on their preferences and experiences on the street views.
% This can easily lead to inconsistent evaluation due to the differences among planners.
$SR$ emphasized the importance of quantitative measurements to evidence-based urban design.
Here, given that an AOI/street can contain vast amounts of street views, planners should analyze the statistics of identified features, including correlations between features, distributions, and standard deviations (T.2.1).
Filtering street views against the values of a specific feature is also important (T.2.2).

\vspace*{-1.5mm}
\item \textbf{T3. Effective Ranking and Comparison:}
% The capability to compare human-scale urban forms in two AOIs/streets is important, as planners can leverage their domain knowledge to gain deep insights into factors affecting the quality of street space.
To help planners quickly narrow down the exploration scope, features among multiple AOIs/streets should be effectively ranked (T.3.1).
Areas/streets with certain features of high values can be easily discovered for further exploration.
After planners select two AOIs/streets, they need to compare the differences in spatial distributions (T.3.2) and the quantitative measurements (T.3.3) of the urban forms. 
% \qm{In addition, effective ranking methods(T.3.3) for the multi features is also helpful for the planers to explore multiple AOIs/streets and narrow down the research scope.}  

\end{itemize}