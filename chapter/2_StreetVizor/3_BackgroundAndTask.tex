\section{Background and Analytical Tasks}
In this section, we introduce our research background and summarize the desired analytical tasks.

\subsection{Background}
\label{sec:c1_bg}

Although the concept of human-scale urban form has only been recently defined~\cite{long_2016_human-scale}, its discussions in the context of urban planning has a long history that can be traced back to the 1960s.
A series of pioneering studies~\cite{jacobs_1961_life, gehl_1971_life} claimed the positive effects of understanding human-scale urban forms in designing high-quality urban space.
Visible human-scale urban forms are particularly important as human beings tend to pay most attentions to surroundings that can be directly seen~\cite{gehl_1971_life}. 

Over the past 10 months, we closely worked with a senior researcher ($SR$) in the field of evidence-based urban design $-$ an emerging research topic in urban planning and design.
$SR$ pointed out that though urban planners have begun to realize the importance and usefulness of street views in analyzing visible urban forms (e.g.,~\cite{rundle_2011_using, li_2015_accessing, naik2014streetscore}), systematic and efficient methods that can facilitate exploration remain lacking.
Hence, $SR$ proposed the development of an efficient visual analytics tool for exploring human-scale urban forms based on GSV images.

To better understand the problem domain, we conducted several rounds of structured interviews with $SR$.
The main analysis criteria are summarized below:

\vspace*{-2mm}
\begin{itemize}

\item
\textbf{Multivariate Features.}
As images contain rich information on the urban environment, the first step is to identify the urban forms for analysis.
Here, we identify five key features that can reflect the quality and livability of street spaces~\cite{jacobs_1961_life, gehl_1971_life}, i.e., $greenery$, $sky$, $building$, $road$, and $vehicle$ features.
$Greenery$ reflects the pleasing greenery view of a street; 
$sky$ and $building$ are correlated with the sense of street closure negatively and positively, respectively; 
and increments in $road$ and $vehicle$ ratios decrease the willingness of people to walk and street attractiveness.

\vspace*{-1mm}
\item
\textbf{Street View Crawling.}
To reveal the surrounding scenes of a street space, street views have to be crawled appropriately: successive images should reflect the continuous change in surrounding scenes.
Hence, the distance between two successive views should not exceed a limit that produces discontinuous scenes; meanwhile, it should not be too small, which will cause computing overload. 
After experimenting with several options, we find 50 meters is a suitable value for the distance between two successive views.

\vspace*{-1mm}
\item
\textbf{Street View Directions.}
Although GSV~\cite{gsv_api} provides 360-degree panorama imagery, only the front and back images in the directions of street headings at sampling locations are required. 
Side views are not utilized because of the following considerations: 
First, side views mainly present building facades and street sides and thus cannot correctly reflect other key features of street space, e.g., $road$.
Second, side views are partially contained by the front and back images at nearby sampling locations. 

\end{itemize}

\vspace*{-1mm}
To evaluate the effectiveness of our approach, we first experiment with a few representative cities.
$SR$ suggested Hong Kong, Singapore, Greater London, and New York City:
Hong Kong and Singapore are dense cities with high-rise buildings in Asia.
Greater London and New York City are well-planned cities in Europe and the US.

%===============================================
\subsection{Analytical Tasks}
\label{ssec:c1_tasks}

After identifying the analysis criteria, $SR$ further raised a list of questions for our system to address, including:
\textit{How are the identified features distributed in an AOI?
What are the feature differences between two AOIs?
What are the exact views that people can see on a street?
Are there any representative views?}

Based on these questions, we compile a list of analytical tasks:

\begin{enumerate}[label={T.\arabic*:}]
	
	\item
	\textbf{Efficient Multi-scale Exploration:}
	Human-scale urban forms are associated with street views at different locations that can be organized on city-, region- and street-scales.
	Planners first need an intuitive overview of the identified feature distributions within a city or a region (T.1.1).
	Next, planners need to explore the details of the urban forms, such as the exact street views, at street level (T.1.2).
	Effective interactions are required to assist users in navigating across different scales.
	
	\vspace*{-1.5mm}
	\item
	\textbf{Quantitative Measurements:}
	% Traditionally, planners usually evaluate a street space qualitatively based on their preferences and experiences on the street views.
	% This can easily lead to inconsistent evaluation due to the differences among planners.
	$SR$ emphasized the importance of quantitative measurements to evidence-based urban design.
	Here, given that an AOI/street can contain vast amounts of street views, planners should analyze the statistics of identified features, including correlations between features, distributions, and standard deviations (T.2.1).
	Filtering street views against the values of a specific feature is also important (T.2.2).
	
	\vspace*{-1.5mm}
	\item
	\textbf{Effective Ranking and Comparison:}
	% The capability to compare human-scale urban forms in two AOIs/streets is important, as planners can leverage their domain knowledge to gain deep insights into factors affecting the quality of street space.
	To help planners quickly narrow down the exploration scope, features among multiple AOIs/streets should be effectively ranked (T.3.1).
	Areas/streets with certain features of high values can be easily discovered for further exploration.
	After planners select two AOIs/streets, they need to compare the differences in spatial distributions (T.3.2) and the quantitative measurements (T.3.3) of the urban forms. 
	% \qm{In addition, effective ranking methods(T.3.3) for the multi features is also helpful for the planers to explore multiple AOIs/streets and narrow down the research scope.}  
	
\end{enumerate}