\section{Expert Review}

To evaluate the effectiveness of StreetVizor, we conducted expert interviews with two independent domain experts other than our collaborating senior researcher $SR$.
One of them focuses on designing livable public space (denoted as $EA$), and the other is an urban ecologist aiming at improving greenery in cities (denoted as $EB$).
Hence, the experts are from different backgrounds: $SR$ and $EA$ in urban planning, while $EB$ in ecology.
In the interviews, we started with explaining the visual encodings and interface design, and demonstrated to them how our system works.
Then, we showed them the case studies, and allowed them to explore the system by themselves for about twenty minutes in the end.
In general, both experts agrees that the way we study human-scale urban forms with street view images is a promising direction.
Their detailed feedbacks are summarized below.

\vspace*{2mm}
\noindent
\textit{Methodology and Approach}.
$EA$ agreed with $SR$ that evidence-based urban design is becoming a trend in urban planning.
He commented, ``StreetVizor is far more than simply a visualization platform. 
Rather, it is an excellent combination of machine learning and visualization techniques, together with classical urban design theories in place-making''.
$EB$ also expressed that ``street view images as an emerging data source can reflect urban environments well", and a visual analytics system can greatly facilitate the exploration of street views.

\vspace*{2mm}
\noindent
\textit{Interactive Visual Design}.
Both experts confirmed that StreetVizor is nicely designed according to the problem domain.
They appreciated the visual consistency across different views.
$EA$ highlighted ``it is very important to use the same colors in different views". 
$SR$ agreed the workflow of ranking multiple AOIs/streets with Ranking Explorer, and comparing two AOIs/streets for details through AOI and Street Explorer is helpful.
``It is easier for me to identify interested regions, such as those in Hong Kong (Fig.~\ref{fig:c1_region-funcitonality})", commented by $SR$.
% They agreed the workflow of over-viewing human-scale urban forms at city- and region-scale first, and examining details at street-scale is helpful.
The AOI Explorer is nicely designed with intuitive map and statistic views.
In particular, ``the statistic view seamlessly integrates three easily understandable plots'', commented by $EB$.
By referring to the scatterplot matrix in Fig.~\ref{fig:c1_statistic_view} \& Fig.~\ref{fig:c1_study_1_statistic}, $EA$ was excited to see the negative correlation between $greenery$ and $building$ $-$ ``there are much space to improve".
$EA$ also liked the visual comparison of human-scale urban form distributions over space (Fig.~\ref{fig:c1_study_1_spatial}), as it reflects the differences of urbanization process and master plans in different cities.

The experts thought presenting street view images in the Street Explorer is intuitive, and mouse hover over to show tree map is helpful.
In contrary, it was difficult in the beginning for both experts to understand the PCP enhanced with street layout information, especially the themeriver plot.
But after exploration, they agreed it is an excellent idea, as it clearly reveals the feature distributions along street.
``Though not common, I believe there are many applications and potentials for the enhanced PCP'', commented by $EB$. 

\vspace*{2mm}
\noindent
\textit{Applicability}.
Both domain experts would like to apply our system to deal with practical problems in their domains. 
Expertized in urban planning and design, $EA$ emphasized ``the lack of efficient tools for human-scale management and design obstacles creating high-quality urban streets.''
 StreetVizor has a great potential to be employed by planners and designers to ``build more pedestrian-oriented and livable streets.'' 
$EA$ also commented ``StreetVizor is highly applicable for evaluating case studies in urban planning''. 
For example, planners can select several key areas, e.g., CBDs, among different cities and then compare their spatial features to develop appropriate strategies in urban renewal.
$EB$ would like to apply our system in environment monitoring, since ``the large amount of GSV images can provide rich information of urban environment''.
He suggested to extend our system in exploring a time-series street view dataset, so that it would allow him to monitor environment changes.

\vspace*{2mm}
\noindent
\textit{Limitations and Improvements}.
The experts pointed out some limitations in our system.
In this work, we explore only six features of human-scale urban forms.
Both experts suggested to extract more urban forms, such as aesthetic amenities and mental well-being, from street view images.
This will advance our system's analytical capabilities and extend its applicabilities.
For instance, a recent study~\cite{jiang_2014_dose} shows that certain relationship may exist between street greenery \& sky ratio and the risk of health challenges.
Besides, they proposed to improve our visual designs.
$EB$ noticed the feature histogram bars in AOI and Street Statistic View are designed differently: one in horizontal, and the other one in vertical style.
He felt the vertical histogram bars are confusing, as they ``do not fit our work habits''.
$EA$ suggested the Street Statistic View can be further improved, by ``encoding neighboring street layouts in the plot'', to reflect spatial information better.
