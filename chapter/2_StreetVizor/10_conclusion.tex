\section{Conclusion and Future Work}

In this chapter, we introduce StreetVizor, a visual analytics system for the exploration of human-scale urban forms based on GSV images.
Through discussions with a collaborating researcher who specializes in evidence-based urban planning, we identify various analysis criteria and formulate a set of analytical tasks.
We integrate some well-estimated visualization techniques, such as juxtaposition map views, scatterplot matrix, and small multiples, into our system.
Specifically, we design an enhanced parallel coordinates with street layout to present coordinated feature values and reveal feature distributions along a street layout.
StreetVizor is used to analyze $\sim$1.7 million street views from Hong Kong, Singapore, Greater London, and New York City.
Using our system, domain experts detect some interesting patterns, such as the negative correlation between $greenery$ and $building$ features.
The experts also agree that our system has a wide range of applications, in areas like public space design and  street environment monitoring.

There are several promising directions for future work.
First, we would like to extract more high-level information, such as signboards, from GSV images.
The signboards can then be compared with points-of-interest information extracted from other data sources, e.g., Foursquare or Google Places.
We anticipate revealing some interesting patterns by fusing multiple types of big urban data.
In addition, to address the scalability issues, we plan to develop more advanced data structures (e.g.,~\cite{lins_2013_nanocubes, wang_2017_gaussian}) to improve the querying and filtering processes in our method.
Besides, we would like to explore new visual interfaces to compare detailed statistics of human-scale urban forms across multiple AOIs/streets at the same time.
% In addition, we plan to adopt network-based distance measurement approach, instead of currently employed Euclidean distance, to cluster the human-scale urban forms.
% We expect to form more representative clusters of street views integrating street network information.
