\section{Discussion}
\label{sec:c1_discussion}

In this work, we combine automatic machine learning and interactive visual analytics techniques to explore human-scale urban forms.
The combination of methods tackles the challenges of integrating information from multiple perspectives and at different scales for analysis.
This approach is attractive for urban planners~\cite{liu_2015_understanding, long_2017_how} because it shows the possibility of transferring traditional subjective and intuitive-oriented urban design to evidence-based and big-data informed methods.
%The case studies show that our approach effectively assist domain experts in finding the street view patterns of interest.

The analysis of human-scale urban form in this work relies heavily on a deep learning technique for image classification.
Therefore, classification accuracy poses serious challenges in our approach.
We select SegNet, which achieves a global accuracy of 82.8\%, out of the different classification techniques that we tried.
The case studies show that the classification technique could provide reasonable analytical results on city- and region-scale human-scale urban form patterns.
Nonetheless, the results remain unsatisfactory in many cases, especially when users would like to examine fine detailed urban forms at street-scale.
We envision applying a more advanced classification algorithm in the near future, given the rapid evolution of image classification techniques.
In addition, the classification is preprocessed offline.
Our system does not support the analysis of street views queried on runtime. 
Thus, its applicability is limited and domain knowledge of planners are underutilized in exploring street views in other cities.
This deficiency can also be resolved with advancement in image classification algorithms and machine computing capabilities.

Moreover, we expect that our system will face scalability issues when the number of street view images increase (e.g. when analyzing street views in more cities). 
To tackle this problem, we can integrate more advanced data structures, such as nanocubes~\cite{lins2013nanocubes} or Gaussian cubes~\cite{wang2016gaussian} for spatial data querying.
More levels of detail and abstraction can also be introduced to handle this problem.
The increase in image number will also burden street view clustering in the interactive exploration process.
We anticipate that certain pre-configurations will facilitate this process.

Presenting multivariate data with spatial information is challenging.
We tackle this challenge by integrating popular PCP with a themeriver plot along an adjusted street layout.
The case studies and feedbacks from experts demonstrate the effectiveness of this design.
Nonetheless, there are some issues with our design.
First, it represents street layout as a simple spline, which is not sufficiently intuitive when the street is straight.
We plan to encode more semantic labels, such as neighboring streets, in the design, as suggested by $EA$.
Second, although the majority of the streets (over 95\%) that we have explored can be rotated and fitted in the rendering space, adjustment does not work in some cases.
Typical examples are streets in a spiral layout.
A more general approach should be developed that can reveal the street layout intuitively and seamlessly fit the street layout in the rendering space.
