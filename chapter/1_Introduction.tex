\chapter{Introduction}\label{chap:intro}

\section{Motivation}
The ongoing urbanization process has been one of the most important trends after World War II . According to the recent urbanization report from the United Nations [17], between 1950 and 2018, the estimated urban population has increased more than fourfold, from 0.8 billion to 4.2 billion. The rapid urbanization has also resulted in a series of problems, such as the traffic congestion and environmental pollution, which attract increased attention from the research field. Traditionally, researchers and urban planners will take tremendous effort to understand and explore large scale urban dynamics due to limited resources and existing cases.

Fortunately, the rapid development of information and communications technology has made urban data collection and analysis cheaper and easier, thus providing opportunities for people to study problems in urban areas and boost the development of relative disciplines, such as urban informatics [9] and urban computing [30]. For example, street view images allow urban planners to conduct urban environment auditing without going to the field. Positional data tracking and acquisition techniques embedded in mobile devices allow researchers to capture the patterns of crowd movement among urban residents. Industrial emissions and meteorological data collected from the monitoring stations help domain experts forecast the air quality accurately and provide control strategies to the government.

Although computers have a great advantage in fast computing and large information storage, fully automated techniques still have limitations in analyzing urban data due to the complexity and variety of analytical tasks in the real world. Final decisions in analyzing data require human intervention, as humans possess acute perception and considerable experience in the domain field. Visualization, which is defined as “the study of transforming data, information, and knowledge into interactive visual representations”  [15], bridges the gap between humans and computing techniques.

In this chapter, we first introduce the existing visualization techniques in urban data exploration. Then, we discuss the major contributions of this proposal, followed by the proposal organization.

\section{Visualization meets urban information}

% The study of urban information leverages the theories from variety of academic fields. 
 
The analysis of model urban always refers to the intersection of three domains: people, place and technology~\cite{foth2011urban}, which is covered by the proposed work in this proposal. 
\begin{itemize}
\item \textbf{People.} People refers to the residents, citizens, community groups as well as the relative data such as commuting, social network, etc.
\item \textbf{Place.} Place refers to the physical environment such as the urban sites, locales and habitats which if formed by nature or human activities.
\item \textbf{Technology.} Technology refers to a variety of techniques span informatics, computing techniques, wearable devices, etc.
\end{itemize}

Before introducing the contribution of this proposal, we first briefly introduce the existing techniques from these three aspects:

\subsection{Visualization of human movement (People)}
As a general topic in visualization discipline, how to effectively visualize the human movement has been long studied in recent years.  An overview has been proposed to categorized the existing movement visualization techniques into three classes: \textbf{pattern extraction}, \textbf{direct depiction}, \textbf{summarization} ~\cite{andrienko2013visual, wu2015telcovis}.

The \textbf{pattern extraction} methods leverage effective visualization to discover the mobility pattern from the movement data such as the traffic congestions~\cite{wang2013visual}, route assessment~\cite{wang2014visual, huang2015trajgraph}, commuting patterns~\cite{beecham2014studying, von2015mobilitygraphs}, co-occurrence~\cite{wu2015telcovis, ni2017spatio}. Compare to other two types of techniques, the pattern extract methods aim to solve the specific application problem directly and these techniques are always difficult to extend to other applications.

The \textbf{direct depiction} directly plot the movement data according to the data format: such as the line segments for OD trails, and polylines for trajectories~\cite{andrienko2013visual, ferreira2013visual, kruger2013trajectorylenses}. Such techniques preserve all the information of individual movement. However, when the dataset is large, the visual clutter the render workload will be the major limitation which hinders the analysts’ ability to explore the data.

The \textbf{summarization} techniques aim to provide an overview of movement data. Most of these summarization techniques conduct statistical calculation and aggregation before rendering. The typical visualization can be heatmap~\cite{wilkinson2009history}, density map~\cite{lanir2014visualizing}, flow map~\cite{guo2014origin}, matrix~\cite{wood2010visualisation} and node-link~\cite{von2016mobilitygraphs}. Due to the ability to present the high-level perception and alleviate the data uncertainty~\cite{andrienko2013visual}, the summarization techniques are widely used in movement visualization. 

\subsection{Visual assist urban environment exploration (Place)}
The rapid urbanization cause a series of changes of living environment including the pollutant and urban forms. Visualization techniques has been used in these domain to discover the potential problems and support the decision making. Since the different application have significantly different analysis requirement, most of the existing work target at specific tasks.  

By combining the data mining and cluster visualization, HydroQual~\cite{accorsi2014hydroqual} is proposed to support a visual analysis of river quality. 
To reason the air pollutant in Hong Kong, Qu et al. ~\cite{qu2007visual} propose a visual analytics system consists of several novel design including  circular pixel bar chart and a parallel coordinates with S-shape axis.  Further more, Deng et al.~\cite{deng2019airvis} propose a visual analytics system assisting the air quality analysts in exploring the air pollutant propagation patterns. 

Visual analytics also enable the urban planers to design and evaluate the city construction. For example, Ferreira et al. ~\cite{ferreira2015urbane} propose Urbane, a 3D multi-resolution framework, which enable the urban planers to conduct the urban development in a data-driven way. Zeng et al.~\cite{zeng2018vitalvizor} proposal a tool to facilitates the urban vitality. Such tool presents both of the physical entities and urban design metrics and allow the expert to quickly discover the city blocks which could be improved.
\subsection{Visual interpretation for techniques (Technology)}
The computing techniques play an important role in the exploration of urban information. Even though more and more techniques are adapted to urban data exploration, most of them are sill used as a black box. For most of the technique developers and users, the interpretability is an important property of the techniques in a variety of real world applications. The interpretation can help the technique developers to improve the technique performance and help the technique users improve their confidence~\cite{strobelt2018lstmvis}. 

The technique interpretation can be classified into two categories: model reduction and feature contribution.
Model reduction methods usually learn a surrogate model to approximate the original complex model.
The surrogate model is usually simple and interpretable, such as linear regression~\cite{ribeiro2016should} and decision trees~\cite{craven1996extracting}. 
Depending on the ways of approximating the original model's behaviors, there are three main ways to conduct model reduction: decompositional, pedagogical, and eclectic~\cite{andrews1995survey}.
% Decompositional methods are usually model dependent and simplify the original model structure, such as the layer and weights of the neural network.
% Pedagogical methods only utilize the input and output information to mimic the original model.
% Eclectic methods are either a combination of the previous two approaches or are distinctively different from them.
% Though model-reduction-based methods are flexible and easy to understand, it is questionable whether or when the surrogate model truly reflects the original model's behaviors.
% We thus discard this approach in our work.

Feature contribution methods help users understand the relationships between input features and the output prediction.
They usually assign each feature an importance score to indicate how it impacts the final prediction.
One classical work is Partial Dependence Plot (PDP)~\cite{friedman2001greedy}, which depicts how feature value changes affect predictions.
A PDP is usually visualized as a line \textbf{}chart in which the x-axis represents feature values and the y-axis represents prediction possibilities (partial dependence scores).
For each feature, its partial dependence scores are usually calculated by iteratively fixing the feature input of all the data points to a certain value and getting the average prediction.
One recent work, SHAP~\cite{lundberg2017unified}, also calculates feature attribution, but from a local perspective.


Even though variety of techniques has been proposed to interpret the mechanism of the complex techniques, very few of them target at the application of the urban information exploration which always take large amount, high-dimensioanl and complex data as input.  

\section{Contribution}
In this proposal, we introduce several visualization techniques in assisting urban data exploration tasks as well as the understanding of urban computing techniques:

\begin{itemize}[noitemsep]
	\item \textbf{Visual Exploration of Human-Scale Urban Forms}. We propose StreetVizor, an interactive visual analytics system that helps planners leverage their domain knowledge in exploring human-scale urban forms based on street view images. Our system presents two-stage visual exploration: 1) an AOI Explorer for the visual comparison of spatial distributions and quantitative measurements in two areas-of-interest at city- and region-scales; 2) and a Street Explorer with a novel parallel coordinate plot for the exploration of the fine-grained details of the urban forms at the street-scale. We integrate visualization techniques with machine learning models to facilitate the detection of street view patterns. 
	\item \textbf{An edge bundling technique for visualizing origin-destination trails in urban traffic.} We introduce RAEB(route aware edge bundling), a novel edge bundling technique to viusally summary the OD trails in urban traffic data. We identify inconsiderate settings of conventional kernel density estimation edge bundling (KDEEB) when applied to urban traffic data, including non-optimal kernel size and road neglect. The limitations are addressed in RAEB by introduction of a comprehensive pipeline comprising preprocessing, bundling and evaluation processes. A series of new parameters, together with adaptions of existing ones, are employed in the pipeline. 
	\item \textbf{Understanding recurrent neural networks on multi-dimensional time-series forecast.} We propose MultiRNNExplorer, a visual analytics system to interpret RNNs on multi-dimensional time-series forecasts.  
	Specifically, to provide an overview to reveal the model mechanism, we propose a technique to estimate the hidden unit response by measuring how different feature selections affect the hidden unit output distribution. 
	We then cluster the hidden units and features based on the response embedding vectors. 
	Finally, we propose a visual analytics system which allows users to visually explore the model behavior from the global and individual levels.
\end{itemize}


\section{Proposal organization}

%The thesis consists of \QMT{three work}, the first two works are proposed to support the urban data exploration, the third work aim to interpret the model 

The rest of this proposal is organized as follows:

Chapter 2 introduces the design of Streetvizor: specific analysis requirements are described by a collaborating urban planner, and the designs are evaluated and refined against requirements. We illustrate the applicability of our approach with case studies on the real-world datasets of four cities, i.e., Hong Kong, Singapore, Greater London and New York City. Interviews with domain experts demonstrate the effectiveness of our system in facilitating various analytical tasks.

Chapter 3 presents the details about the implementation of RAEB(route aware edge bundling), a new technique to visualize the origin-destination trails in traffic data by edge bundlings techniques. In the evaluation, we conducted experiments with artificial and real-world traffic data to demonstrate the advancements of RAEB in various applications, including the preservation of road network topology and support of multi-scale exploration. 

Chapter 4 proposes a MultiRNNExplorer, a visual analytics system to interpret the recurrent neural networks on multi-dimensional time-series forecast. We introduce the system design and demonstrate the effectiveness of the proposed technique with case studies of air pollutant forecast applications.


Chapter 5 concludes the proposal and discusses possible future work.