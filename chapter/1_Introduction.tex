\chapter{Introduction}\label{chap:intro}

\section{Motivation}
The ongoing urbanization process has been one of the most important trend after the World War II (WWII). According to the recent urbanization report from United Nations~\cite{error}, between 1950 and 2018, the estimated urban population increased more than fourfold, from 0.8 billion to 4.2 billion. The rapid urbanization also result in a series of problems such as the traffic congestion and environmental pollution, \QMT{which attracts more and more attentions from the research field.} 

Traditionally, it will take tremendous effort for researchers and urban planers to understand and explore the large scale urban dynamics due to the limited resource and existing cases. 
Fortunately, the rapid development of information and communication technology(ICT) makes the urban data collection and analysis cheaper and easier, which provides the opportunities for people to understand, forecast and even solve the problems. For example, the street view images allow urban planners to conduct urban environment auditing without going to the field; the position data tracking and acquisition techniques \QMT{embedded} in the mobile devices allow researchers to capture the crowd movement pattern of urban residents; industrial emission and the meteorology data collected from the monitoring stations is able to help domain expert in precisely forecasting the air quality in future and \QMT{provide} the control strategy to the government. 
All of these boost the development of urban computing discipline.  

Even though the computers have a great advantage in fast computing and large information storage, the fully automatics techniques still have limitations in analyzing the urban data due to the complexity and variety of the real world analysis tasks.  It is always need the involvement of human beings who have acute perception and considerable experience in the domain field to make the final decision. Visualization, "the study of transforming data, information, and knowledge into interactive visual representations"~\cite{liu2014survey}, bridges the gap between the human beings and the computing techniques.

In this chapter, we first introduce urban informatics and the existing visualization techniques in urban data exploration. Then we discuss the major contributions in this thesis which is followed by the thesis organization.

\section{Visualization meets urban informatics}

\section{Contribution}

\section{Thesis organization}