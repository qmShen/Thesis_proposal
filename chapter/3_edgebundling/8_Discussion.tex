\section{Discussion}
\label{sec:discussion}

The experiments demonstrate that RAEB outperforms KDEEB in visualizing OD trails in urban traffic data.
The first experiment with artificial data shows that the introduction of route awareness $p_{ra}$ in RAEB distinguishes grid and hierarchical network structures.
The second experiment presents the comparisons between KDEEB and RAEB using New York taxi data, which records only origin and destination positions. 
The results with different kernel size $p_{r}$ and in multi-scales indicate that, 
(1) $p_{r}$ should be measured based on not only graph drawing size, but also underlying road network topology;
and (2) RAEB supports better multi-scale exploration.
The robustness of RAEB is further demonstrated through the third experiment with Shenzhen taxi data, which records a sequence of GPS positions in every 20 seconds.

The advancements are achieved by parameter settings adopted in RAEB.
This section presents detailed discussion on the parameters, followed by performance comparison between RAEB and KDEEB. 
In the end, we discuss about generality and limitation issues.

%%%%%%%%%%%%%%%%%%%%%%%%%%%%%%%%%%%%%%%%%%%%%%%%%%%%%%%
\begin{table}[h]
	\begin{tabularx}{0.995\textwidth}{|>{\hsize=0.3\hsize}X|
			>{\hsize=0.18\hsize}X|
			>{\hsize=0.52\hsize}X|}
		\hline
		\textbf{Parameters} & \textbf{KDEEB} & \textbf{RAEB} \\
		\hline \hline
		Route awareness ($p_{ra}$)& $-$ & 0 - Number of road hierarchies\\
		\hline
		Kernel size ($p_r$) & Graph drawing & Graph drawing and road network geometry \\
		\hline
		Bundling iterations ($p_n$)& 10 - 15 & $-$ \\
		\hline
		Bundling stability ($p_{s}$)& $-$ & \textit{NMI} between images in consecutive iterations\\		
		\hline
		Bundling deviation ($p_{d}$) & $-$ & of bundles to OD trails\\
		\hline
	\end{tabularx}
	\caption{Main parameter adaptions in RAEB, in comparison with these in KDEEB.}
	\label{tab:parameters}
	\vspace{-1mm}
\end{table}

\subsection{Parameters}
\label{ssec:parameters}

RAEB adapts and introduces new parameters to better visualize OD trails in urban traffic data.
Table~\ref{tab:parameters} presents an overview of these parameter settings in comparison with those in KDEEB.

\begin{itemize}

\item
\textbf{Route awareness ($p_{ra}$):}
An input graph for KDEEB is straight lines connecting end nodes directly.
As discussed in the first experiment with artificial OD trails, such an input graph cannot distinguish grid and hierarchical road networks.
Instead, we introduce route awareness parameter $p_{ra}$ to control layout of input graph.
The first experiment also shows that $p_{ra}$ can be manipulated to generate bundles representing levels of details of road network.
If $p_{ra}$ is set to zero, input graph will be straight lines and the results are the same with KDEEB;
if $p_{ra}$ is set to the highest number of road hierarchies, input graph will be OD trails in line with road network, and more roads can be preserved.

\vspace{1mm}
\item
\textbf{Kernel size ($p_{r}$):}
In the second experiment with New York taxi data, we show that the parameter kernel size $p_r$ can control bundling coarseness.
KDEEB computes $p_{r}$ as 5\% of graph drawing size, which will generate coarse bundles that can only reveal main traffic connections (Figure~\ref{fig:nyc_visual}(b)).
Instead, RAEB suggests $p_{r}$ should be measured based on both graph drawing size and road network geometry.
In this way, a smaller $p_{r}$ is generated, and fine bundles that reveal details of street networks are generated (Figure~\ref{fig:nyc_visual}(c) \& (d)).
The advantage of our approach is further demonstrated using Shenzhen taxi data, which exhibits imbalanced distribution of taxi trips in the city.

\vspace{1mm}
\item
\textbf{Bundling iteration ($p_{n}$):}
KDEEB presets bundling iteration $p_{n}$ to a fixed number in between 10 - 15, such that to terminate the bundling process.
The choice of a suitable $p_{n}$, however, needs much experience and varies among different people. 
Adjustments in other parameters may also require fitting of the parameter.
RAEB discards this subjective parameter.

\vspace{1mm}
\item
\textbf{Bundling stability $p_{s}$:}
Instead, RAEB introduces a new parameter $-$ bundling stability $p_{s}$, to control bundling process termination.
$p_{s}$ is measured as normalized mutual information (\textit{NMI}) between images generated in two consecutive bundling iterations.
\textit{NMI} is a common metric in image processing field, thus can be easily integrated.
When $p_{s}$ reaches a predefined threshold, which means that visually a stable structure is generated, the bundling process will be automatically terminated.


\vspace{1mm}
\item
\textbf{Bundling deviation $p_{d}$:}
While $p_{s}$ measures bundling stability of a bundling method, it still lacks a method to quantitatively compare multiple ones.
The parameter bundling deviation $p_{d}$ is introduced to fill the gap.
$p_{d}$ is measured as average Frechet distance of bundles to corresponding OD trails.
The measurement is widely employed in geographical science, and we compare KDEEB and RAEB regarding the parameter below.

\end{itemize}

Some of these parameters can be automatically computed, including $p_r$ and $p_d$, while settings of $p_{ra}$ and $p_{s}$ can be consistent across different input data.
From the experiments, we recommend 1 for $p_{ra}$ (corresponding to top 5\% ranked routes), and 75\% for $p_{s}$.

%%%%%%%%%%%%%%%%%%%%%%%%%%%%%%%%%%%%%%%%%%%%%%%%%%%%%%%
\subsection{Performance}

\begin{table}[!tb]
	\setlength\extrarowheight{2pt}
	\begin{tabularx}{0.95\textwidth}{|
							  >{\hsize=0.19\hsize}X|
                              >{\hsize=0.16\hsize}X|
                              >{\hsize=0.05\hsize}X|
                              >{\hsize=0.17\hsize}X|
                              >{\hsize=0.13\hsize}X|
                              >{\hsize=0.17\hsize}X|
                              >{\hsize=0.13\hsize}X|}
	\hline
	Graph & Edge & $p_n$ & \multicolumn{2}{c}{Time (sec.)} \vline & \multicolumn{2}{c}{Deviation (pixel)} \vline \\
	& Samples & & KDEEB & RAEB& KDEEB & RAEB \\
	\hline \hline
	Random (grid) & 4.6M & 13 & 40.3 & 50.7 & 18.37 & 12.58 \\ 
	\hline
	NewYork & 3.1M & 11 & 34.3 & 42.9 & 15.40 & 9.88 \\
	\hline
	Shenzhen & 1.3M & 8 & 13.8 & 22.8 & 13.71 & 10.53 \\
	\hline
	\end{tabularx}
	\caption{Statistics comparison of KDEEB and RAEB for datasets used in the experiments. }
	\label{tab:statistics}
\vspace{-1mm}
\end{table}

Table~\ref{tab:statistics} shows comparison of KDEEB and RAEB for the datasets used in the experiments.
Both methods are implemented in Java running on a MacBook Pro with a 3.1 GHz Core i7 and Radeon Pro 560 graphics.
We employ the same kernel size $p_r$ and bundling iterations $p_n$ measured by our approach for both methods.

KDEEB's bundling process include sampling, splatting, advecting and smoothing steps.
Bundling time for each iteration is mainly determined by number of edge samples and kernel size.
Since kernel size measured by our approach is generally smaller than that by KDEEB, bundling time is reduced. 
The total running times for KDEEB methods are comparable with those in the original work~\cite{hurter2012graph}.
RAEB requires an additional step of bundling stability measurement, which is determined by the graph drawing size.
This adds in a running time of less than 1 seconds in each iteration for all three experiments.
The measurement is computed in every bundling iteration, thus we can shorten time by reducing the measurement frequency.
All the steps can be further accelerated using GPU e.g. CUDA~\cite{van2016cubu}.

On the other side, bundles generated by our method are more close to OD trials, as revealed by comparison regarding bundling deviation. 
From the table, RAEB reduces about 1/3 deviations of those in KDEEB for artificial and New York taxi datasets. 
The improvement is smaller for Shenzhen taxi data, though graph drawing in Shenzhen experiment is bigger.
This is probably because taxi trips are concentrated in a small area in the south of Shenzhen.
Nevertheless, the result image by RAEB shows more improvements than that by KDEEB (Figure~\ref{fig:shenzhen}(c) vs (b)). 


%%%%%%%%%%%%%%%%%%%%%%%%%%%%%%%%%%%%%%%%%%%%%%%%%%%%%%%
\subsection{Applicability and Limitations}
\label{ssec:app}

The experiments have shown the applicability of RAEB for bundling and visualizing OD trails in urban traffic using real-world taxi data.
RAEB introduces new parameters of route awareness $p_{ra}$ and bundling deviation $p_d$, and adapts existing measurement of kernel size $p_r$, to better fit road network topology.
The approach can be extended to other types of urban traffic data, such as commuter trips using public transportation, or people traces derived from call detail records.
These dataset consists of both OD trails and underlying road network.
Through advanced map matching algorithms, urban traffic's real paths on a road network can be recovered.
However, the parameters are not applicable to general graphs with information of only node connections, except for bundling stability $p_{bs}$.

Edge bundling methods face a common issue of unrealistic bundle lines.
Specifically for OD trails in urban traffic, bundle lines should not be far away from real paths.
RAEB mitigates the issue with route awareness $p_{ra}$ for controlling edge layout of input graph, and bundling stability $p_{bs}$ for generating stable results.
The first experiment shows that $p_{ra}$ can preserve certain routes on demand, while the second experiment reveals that RAEB supports better multi-scale exploration.
Experimental results of bundling deviations $p_{d}$ further proves the advantage of RAEB over KDEEB.
Nevertheless, these parameters can only constrain bundle layout in certain extent, as resulting bundle lines are still displaced from OD trials.
In scenarios requiring precise positions, such as query movements passing through a waypoint~\cite{kruger_trajectorylenses_2013} or road segment~\cite{wang_2014_visual-reasoning}, visual hints are required to remind the displacement.

Bundles generated by RAEB can show clear traffic patterns, meanwhile preserve road network topology better than KDEEB in a city.
These main goals of this work have been successfully achieved, as demonstrated by the experiments.
On the other hand, RAEB is not able to reveal detailed traffic patterns such as directional movements.
Direction is of recognized importance for studying OD trails in urban traffic~\cite{zeng_2015_visualizing}.
This limitation can be addressed by including directional information in density estimation or edge advecting functions, as described in CUBu~\cite{van2016cubu}.
Besides, RAEB is an image-based edge bundling method, constraining its applicability for 2D rendering only.
Applications such as to bundle movements~\cite{lambert20103d, thony2015vector} and fiber tracts~\cite{hurter_2018_functional} in 3D, and graphs in virtual reality~\cite{kwon_2016_study}, need spring- or geometry-based bundling methods.