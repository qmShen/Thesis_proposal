\section{Conclusion and future work}
\label{sec:con}

OD visualization is challenging $-$ no universally good technique is available for visualizing arbitrary ODs~\cite{andrienko_visual_2012}.
In particular, this work studies OD trails in urban traffic.
We identify inconsiderate settings of existing kernel density estimation edge bundling (KDEEB) method when applied in the scenario.
This is because KDEEB was developed for general graphs such as airlines and migrations, while ignores data nature of urban traffic data.
These inconsiderate settings can cause domain-specific problems such as no preservation of road network topology and poor support of multi-scale exploration.
We contribute to the field with a new technique, namely route-aware edge bundling (RAEB). 
The bundling process is complemented with preprocessing that generate more proper input edge layout, and evaluation that create more stable visual results.
A series of parameter adaption and introduction are made to better cater to geometric properties of OD trails and road networks.
We compare RAEB with KDEEB in regarding to bundling time and deviation using artificial and real-world taxi data.
The experiment results show that RAEB outperforms KDEEB in reducing bundling deviation meanwhile achieving comparable computation speed.

Even though RAEB is specifically designed for bundling OD trails in urban traffic, some parameters such as bundling stability $p_{bs}$ can be applicable to other iterative edge bundling methods, including SBEB~\cite{ersoy2011skeleton} and FDEB~\cite{holten2009force}.
We would like to try integrating these parameters in the bundling methods, such that to make them more controllable.
We also realize many computation tasks in RAEB can be parallelized, thus bundling efficiency can be improved by GPU acceleration.
It would be worthy to reimplement RAEB in GPU, similar to CUBu~\cite{van2016cubu}.
Besides bundling control and efficiency issues, how to access bundling quality and faithfulness are the main challenges hindering the applicability of edge bundling methods~\cite{lhuillier2017state}.
Our experiences in developing RAEB suggest that in specific application scenarios, these challenges can be solved with domain knowledges.
On the basis of treating OD trails mapped onto road network as ground truth, our proposed parameter of bundling deviation $p_d$ is a quality and faithfulness metric in certain extent.
The parameters is derived from Frechet distance that is a common metric in geographical science.
In the future, we would like to apply RAEB in other fields such as to visualize neural connections in brain~\cite{2014_bottger_three-d, yang_2017_blockwise}.
Adaptions are most likely required, and we envision close collaborations with domain experts can facilitate the development.