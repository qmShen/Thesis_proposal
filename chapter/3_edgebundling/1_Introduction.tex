\section{Introduction}
\label{sec:intro}

Movement can be defined as change of an object's positions or geometric attributes over time~\cite{dodge_2008_towards}.
Within a specific time period, the movement of an object can be modeled with an origin (O), a destination (D), and consecutive positions (trail) in-between.
Due to fast development of location sensing technologies, massive amounts of OD trails, such as vessel movements and taxi trips, have been collected.
Studies of OD trails have revealed many movement patterns and contributed to many applications, e.g. diseases spread control~\cite{brock_2006_scaling} and transportation planning~\cite{wang2012understanding}.

Visualizing OD trails is a hot yet challenging topic.
Conventional method that connects origins to destinations with lines~\cite{tobler_1987_experiments} can easily cause visual clutter.
To tackle the problem, methods of aggregating trails into flows are usually employed.
A number of automatic techniques have emerged, such as intelligent routing layout~\cite{phan2005flow, verbeek_2011_flow} and spatial mapping~\cite{andrienko_spatial_2011-1, guo2014origin}.
These methods are preferable for revealing overall traffic patterns for answering general questions, e.g., what are popular roads in a city? where do people head to?
However, the methods mostly ignores information of individual OD trail.
In certain scenarios such as to find abnormal people movements, additional analytics are required to complement the visualization.

Wrapping up OD trails into bundles can reveal overall traffic patterns, meanwhile keep individual OD trail.
Many bundling methods including spring-based (e.g.,~\cite{holten2009force, selassie2011divided}), image-based (e.g.,~\cite{hurter2012graph, lhuillier2017ffteb}), and geometry-based (e.g.,~\cite{holten2006hierarchical, cui2008geometry}), have been proposed.
These techniques have been successfully employed for many different datasets such as Internet connections and migrations.
In these data, it is flexible to manipulate edges as long as node connections are kept.
This, however, does not hold for OD trails in urban traffic data.
In a city, humans and vehicles are moving along roads, and many activities including traffic jams and accidents, are happening on roads.
Hence, OD trails are constrained by road networks.
Trail information is equally, if not more, important as ODs for urban traffic data.

Among various edge bundling methods, kernel density estimation (KDEEB)~\cite{hurter2012graph} is in state-of-the-art with advantages in speed and generality.
However, we identify several inconsiderate settings of KDEEB when applied to OD trails in urban traffic data, including non-optimal kernel size and road neglect.
These settings wreck applicability of KDEEB for movement visualization, which requires to preserve road network topology and to support multi-scale exploration.
This work presents a new method, namely route-aware edge bundling (RAEB), specifically designed to address the limitations.
RAEB employs a comprehensive pipeline consisting of preprocessing, bundling and evaluation processes.
A series of new parameters such as route awareness and bundling stability, which leverage advanced techniques from geography and image processing fields, are introduced.
Experiments on artificial and real-world urban traffic data show that RAEB can generate more realistic results, meanwhile achieve fast computation speed comparable to KDEEB.

\vspace{1.5mm}
The primary contributions of this work include:

\begin{itemize}

\item
We develop a new edge bundling method of RAEB, which includes a comprehensive pipeline consisting of preprocessing, bundling, and evaluation processes. 

\item
We adapt existing parameters in KDEEB, and introduce a series of new parameters, to better fit OD trails in urban traffic.

\item
We conduct several experiments using artificial and real-world urban traffic data to demonstrate effectiveness of RAEB. 
\end{itemize}