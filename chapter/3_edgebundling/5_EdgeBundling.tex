\section{Bundling Method}
\label{sec:bundling}
\begin{algorithm}[h]
	\caption{KernelSizeMeasurement}\label{al:kernel_size_measurement}
	\begin{algorithmic}[1]
		\Require Top \textit{N} sorted routes as polyline list $\textbf{P} = \{P_1, ..., P_N\}$
		\Ensure Initial kernel size $p_r$		
		\State Let $d$[ ][ ] denote distance between polyline pairs
		\For {i = 1 to $N$}
		\For {j = i + 1 to $N$}
		\State $d$[i][j] = $d$[j][i] = DiscreteFrechetDistance($P_i$, $P_j$)
		\EndFor
		\EndFor
		\State Let \textbf{C} denote a list of polyline clusters;
		\State \textbf{C} = $DBSCAN(\textbf{P}, eps, minNum)$;
		\State $C_{max}$ = $argmax_{|C_i|}(C_i | C_i \in \textbf{C})$;
		\State Initialize $d_{geo} = 0$;
		\State Let $count = |C_{max}| \times (|C_{max}| - 1)$;
		\ForEach {$P_i \in C_{max}$}
		\ForEach {$P_j \in C_{max}$ \&\& $P_i \neq P_j$}
		\State $d_{geo}$ = $d_{geo}$ + $d$[i][j] / count;
		\EndFor
		\EndFor
		\State $p_r$ = ToDrawingSpace($d_{geo}$ / 2); \\
		\Return $p_r$
	\end{algorithmic}
\end{algorithm}


RAEB adapts KDEEB bundling algorithm~\cite{hurter2012graph} to better reveal topology of urban traffic.
This sections presents amendments employed by RAEB, including kernel size measurement and density map generation.



%%%%%%%%%%%%%%%%%%%%%%%%%%%%%%%%%%%%%%%%%%%%%%%%%%%%%%%%%%%%%%%%%%%%%%%%%%%%%%%%%%%%%%%%%%%%%%%%%%
%%%%%%%%%%%%%%%%%%%%%%%%%%%%%%%%%%%%%%%%%%%%%%%%%%%%%%%%%%%%%%%%%%%%%%%%%%%%%%%%%%%%%%%%%%%%%%%%%%
\subsection{Optimal Kernel Size}
\label{ssec:kernel_size}
KDEEB only considers the size of graph drawing when determining kernel size.
This strategy may not be optimal for bundling OD trails in urban traffic (\textit{P1}).
Ideally, the chosen kernel size should be able to bundle closely-related OD trails (e.g., movements on bi-directional roads), while separate loosely-correlated OD trails (e.g., movements on two different highways).

To address \textit{P1}, both graph drawing size and geometric property of road network should be considered.
We develop an automatic process for measuring initial kernel size, as outlined in Algorithm~\ref{al:kernel_size_measurement}.
Here, top $N$ routes are extracted from \textit{Preprocessing} stage.
The routes are treated as a polyline list \textbf{P}, and inputted into the algorithm.
Pair-wise distance between each pair of ploylines are computed. 
Then a DBSCAN algorithm is applied to group the polylines into polyline clusters \textbf{C}.
The cluster with maximum number of polylines is extracted, and denoted as $C_{max}$.
Average geographical distance $d_{geo}$ is computed for polylines in $C_{max}$.
Finally, half of $dist_{geo}$ is converted to $p_r$ in graph drawing space, and $p_r$ is returned as the initial kernel size.

Basically, the algorithm first computes average geographical route distance in the top $N$ routes, and then converts the distance to graph drawing space.
$N$ should not be too small that the routes are not representative for the whole road network; meanwhile, it should not be too big to overlap route awareness.
From our experiments, we find 1\% of whole route size is appropriate for $N$.
Besides, another important factor is the distance metric between two polylines.
Here, we choose discrete Frechet distance, which measures similarity of two polylines considering both locations and ordering of the points along the polylines.
The metric is one of the most popular methods for movement analysis~\cite{gudmundsson2011computational}, and can be computed efficiently~\cite{eiter_1994_computing}.

%%%%%%%%%%%%%%%%%%%%%%%%%%%%%%%%%%%%%%%%%%%%%%%%%%%%%%%%%%%%%%%%%%%%%%%%%%%%%%%%%%%%%%%%%%%%%%%%%%
%%%%%%%%%%%%%%%%%%%%%%%%%%%%%%%%%%%%%%%%%%%%%%%%%%%%%%%%%%%%%%%%%%%%%%%%%%%%%%%%%%%%%%%%%%%%%%%%%%
\subsection{Density Map Generation}
KDEEB omits trail information (\textit{P3}), as it connects ODs with directed lines.
We aim to keep bundles close to OD trails on road network.
More specifically, in \textit{Preprocessing} stage, each OD trail is abstracted into a sequence of artificial directed lines and real routes in road network.
The artificial directed lines are free to manipulated, while we would like to keep the bundle layout spatially close to the real routes.
In addition, it is preferable to keep the bundling paradigm of advecting sampling sites towards high density directions.

To achieve this goal, we modify the density map generation used by KDEEB.
Let denote a list of routes $R_{aware} \subset \mathbb{R}^2$ are kept for route awareness.
Since sampling sites are advected towards their gradient directions, gradients of points along $R_{aware}$ should be either (1) zero, or (2) pointing to route direction.
To minimize effects on densities surrounding $R_{aware}$, we choose the second option.
Instead of Eqn.~\ref{eq:kernel_density_estimation}, we now use

\vspace{-7mm}
\begin{equation}\label{eq:new_density}
\rho_{od} = \rho + D(\textbf{x} | \textbf{x} \in R_{aware})
\end{equation}
\vspace{-1mm}

\noindent
where $D(\textbf{x}) \in \mathbb{R}^+$ is a constant assigned to a point $\textbf{x} \in R_{aware}$.
High $D$ value will make it more likely that gradient at $\textbf{x}$ points to the same direction with the route.