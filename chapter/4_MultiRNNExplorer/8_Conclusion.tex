\section{Discussion and Conclusion}

In this chapter, we presented MultiRNNExplorer, a visual analytic system for understanding RNN models for high-dimensional time-series forecasting.
Specifically, we use air pollutant forecasting as the target application. 
To understand the the model mechanism from a global perspective, we propose a technique to estimate the hidden units' response to an individual feature by measuring how different feature selections affect the hidden units' output distribution. 
From a finer granularity, we further use the gradient-based method to measure the local feature importance for each sequence instance. 
Based on these techniques, we design a visual analytic system which enables the users to explore and reason about the model behavior from different perspectives. 
Our evaluation includes three case studies that demonstrate the effectiveness of the proposed system for comprehensive analysis of RNNs.
Meanwhile, there are some issues need to be discussed:

\textbf{Scalability.}
Several views may suffer scalability issues when the number of cases increase. 
In Projection View, thousands of individual cases need to be visualized and cause serious visual clutter due to the overlap of circles.  
% Even though we provide interactions such as zoom and pan which enable the users to iteratively explore the cases, it is still difficult for the users to clearly observe the overview of the case similarity.  
In our case, more than 8000 points are visualized. 
If data size keeps increasing, we may also apply other advanced projection techniques\cite{van2017visual, pezzotti2016hierarchical} for Projection View. 
The individual view also has such a problem: it is easy for users to brush hundreds of individual cases from Projection View and generate tens of clusters. Due to the limited screen size, our current design allows 9 groups of individual cases to be shown at the same time and uses the scroll bar to enable the observation of more groups.
% Throughout the system, the feature type is encoded by categorical colors. Because humans suffer perception issues for more than 10 categorical colors, we manually choose similar colors for highly correlated features (for example, $PM_{2.5}$ and $PM_{10}$ are encoded by light green and dark green). For more features, we may need only a few colors for the features of interest and encode other features by the same color (for example, gray). 

\textbf{Generalization.}
Though we use air pollutant forecasting as example in this chapter, the proposed method can be easily extended to other high-dimensional time-series forecasts with few changes.
The only design that may need to be fine-tuned to is feature glyphs (Fig.~\ref{fig:cluster_design}).
The current design supports encoding three spatial attributes including direction, type, and distance and more design choices can be explored according to different requirements in other domains. 
% Any other cell-like glyph design is also applicable in our method. 


% There is some promising future work for MultiRNNExplorer. 
There are also some future directions to improve MultiRNNExplorer.
% First, improve the efficiency of the response estimation. With 8375 test cases, the current method needs hours of calculation which has to be done offline. 
% In the future work, we will improve this method by introducing rational sampling techniques or develop alternative estimation techniques to enable our system to support realtime data processing. 
One approach is to improve the individual comparison. 
In our current design, the individual comparison requires comparing data instances side by side. 
Supporting interactions to highlight the differences would further benefit users in this scenario.
% A specific comparison design will be helpful to ease the burden of users. 
We also consider to use our system on other high-dimensional forecasting applications including such as fraud detection\cite{jurgovsky2018sequence}. 
We are also exploring the possibility of applying MultiRNNExplorer to audio modeling tasks. 

