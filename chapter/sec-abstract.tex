\begin{abstract}
The rapid urbanization has been one of the most important global trends in the recent 50 years. Even though half of the world's people living in urban area contribute to 80 percent of GDP(Global Domestic Product), the ever more crowded urban area results in a series of problems such as traffic congestion, environment pollutant, insufficient resource, unbalanced urban infrastructure, etc. Fortunately, the developed sensing technology makes the data collecting and processing easier and cheaper, which provides the opportunity for people to understand the phenomenon or even figure out the solution to address these problems. However, due to the high-dimensionality, heterogeneity of the dataset as well as the complex analytical tasks, the pure automatics techniques are insufficient in the urban information exploration. On the other hand, the human analysts who have sharp perception and domain expertise cannot deal with large volumes of data without powerful tools. Visualization bridges the gap between analysts and auto-techniques, which has been widely applied in urban information exploration. 

In this proposal, we introduce several novel visual analytics techniques which cover the three domains in urban information exploration: place, people and technology. 
In the first work, we propose StreetVizor, a visual analytics system which helps the urban planners to explore the fine-scale living environment. The system automatically extracts the human-scale urban form features from street view images through machine learning techniques. Then the comprehensive analysis framework and novel visual designs are proposed to support the free exploration from multi-levels. 
In the second and third work, we target at visual analytics of massive human movement data. We first propose a RAEB, an edge bundling techniques to visualize the overview of origin-destination trails. By introducing the additional graph structure as constraints, the trail bundles can follow the traffic network in the city. Then, we dig into the contact pattern of human movement. After extracting the contact dataset from movement data, a novel visual design is proposed to support the interactive exploration. 
In the last work, we focus on the model interpretation in the air pollutant forecast application. We propose MultiRNNExplorer, which visualizes the recurrent neuron network behaviors in multi-dimensional time-series forecast. 
To validate the effectiveness of the proposed techniques, we conduct several studies based on real-world datasets and domain expert interview.  

\end{abstract}
