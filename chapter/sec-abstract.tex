\begin{abstract}
Rapid urbanization has become one of the most important global trends in the last 50 years. Although half of the world’s population live in urban areas and contribute to 80 percent of the world’s GDP , the ever more crowded urban areas result in a series of problems, such as traffic congestion, pollution, insufficient resources, and unbalanced urban infrastructure.  Fortunately, the development of sensing technology has made data collection and processing easier and cheaper, thus providing an opportunity for people to understand the phenomenon or even determine the solutions to address these problems. However, due to the high dimensionality, heterogeneity of the dataset, and the complex analytical tasks, the pure automated techniques are insufficient in the exploration of urban information. On the other hand, the human analysts with sharp perception and domain expertise cannot deal with large volumes of data without powerful tools. Visualization bridges the gap between analysts and automated techniques, and it has been widely applied in the exploration of urban information.

In this proposal, we  introduce several novel visual analytics techniques that cover the three domains in urban information exploration: place, people, and technology. In the first work, we propose StreetVizor, a visual analytics system that helps urban planners to explore fine-scale living environments. The system automatically extracts the features of human-scale urban form from street view images through machine learning techniques. Then, a comprehensive analysis framework and novel visual designs are proposed to support free exploration from multiple levels. In the second work, we target the visualization of massive human movement data. We propose route-aware edge bundling, which visualizes the overview of origin–destination trails. By introducing the additional graph structure as constraints, the trail bundles can follow the traffic network in the city. In the last work, we focus on the model interpretation in the application of air pollutant forecast. We propose MultiRNNExplorer, which visualizes the recurrent neuron network behaviors in multi-dimensional time-series forecast. To validate the effectiveness of the proposed techniques, we conduct several studies based on real-world datasets and domain expert interviews.


\end{abstract}
